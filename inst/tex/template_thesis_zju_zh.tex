\documentclass[]{style/zjuthesis}

\usepackage{color}
\usepackage{fancyvrb}
\newcommand{\VerbBar}{|}
\newcommand{\VERB}{\Verb[commandchars=\\\{\}]}
\DefineVerbatimEnvironment{Highlighting}{Verbatim}{commandchars=\\\{\}}
% Add ',fontsize=\small' for more characters per line
\usepackage{framed}
\definecolor{shadecolor}{RGB}{248,248,248}
\newenvironment{Shaded}{\begin{snugshade}}{\end{snugshade}}
\newcommand{\KeywordTok}[1]{\textcolor[rgb]{0.13,0.29,0.53}{\textbf{{#1}}}}
\newcommand{\DataTypeTok}[1]{\textcolor[rgb]{0.13,0.29,0.53}{{#1}}}
\newcommand{\DecValTok}[1]{\textcolor[rgb]{0.00,0.00,0.81}{{#1}}}
\newcommand{\BaseNTok}[1]{\textcolor[rgb]{0.00,0.00,0.81}{{#1}}}
\newcommand{\FloatTok}[1]{\textcolor[rgb]{0.00,0.00,0.81}{{#1}}}
\newcommand{\ConstantTok}[1]{\textcolor[rgb]{0.00,0.00,0.00}{{#1}}}
\newcommand{\CharTok}[1]{\textcolor[rgb]{0.31,0.60,0.02}{{#1}}}
\newcommand{\SpecialCharTok}[1]{\textcolor[rgb]{0.00,0.00,0.00}{{#1}}}
\newcommand{\StringTok}[1]{\textcolor[rgb]{0.31,0.60,0.02}{{#1}}}
\newcommand{\VerbatimStringTok}[1]{\textcolor[rgb]{0.31,0.60,0.02}{{#1}}}
\newcommand{\SpecialStringTok}[1]{\textcolor[rgb]{0.31,0.60,0.02}{{#1}}}
\newcommand{\ImportTok}[1]{{#1}}
\newcommand{\CommentTok}[1]{\textcolor[rgb]{0.56,0.35,0.01}{\textit{{#1}}}}
\newcommand{\DocumentationTok}[1]{\textcolor[rgb]{0.56,0.35,0.01}{\textbf{\textit{{#1}}}}}
\newcommand{\AnnotationTok}[1]{\textcolor[rgb]{0.56,0.35,0.01}{\textbf{\textit{{#1}}}}}
\newcommand{\CommentVarTok}[1]{\textcolor[rgb]{0.56,0.35,0.01}{\textbf{\textit{{#1}}}}}
\newcommand{\OtherTok}[1]{\textcolor[rgb]{0.56,0.35,0.01}{{#1}}}
\newcommand{\FunctionTok}[1]{\textcolor[rgb]{0.00,0.00,0.00}{{#1}}}
\newcommand{\VariableTok}[1]{\textcolor[rgb]{0.00,0.00,0.00}{{#1}}}
\newcommand{\ControlFlowTok}[1]{\textcolor[rgb]{0.13,0.29,0.53}{\textbf{{#1}}}}
\newcommand{\OperatorTok}[1]{\textcolor[rgb]{0.81,0.36,0.00}{\textbf{{#1}}}}
\newcommand{\BuiltInTok}[1]{{#1}}
\newcommand{\ExtensionTok}[1]{{#1}}
\newcommand{\PreprocessorTok}[1]{\textcolor[rgb]{0.56,0.35,0.01}{\textit{{#1}}}}
\newcommand{\AttributeTok}[1]{\textcolor[rgb]{0.77,0.63,0.00}{{#1}}}
\newcommand{\RegionMarkerTok}[1]{{#1}}
\newcommand{\InformationTok}[1]{\textcolor[rgb]{0.56,0.35,0.01}{\textbf{\textit{{#1}}}}}
\newcommand{\WarningTok}[1]{\textcolor[rgb]{0.56,0.35,0.01}{\textbf{\textit{{#1}}}}}
\newcommand{\AlertTok}[1]{\textcolor[rgb]{0.94,0.16,0.16}{{#1}}}
\newcommand{\ErrorTok}[1]{\textcolor[rgb]{0.64,0.00,0.00}{\textbf{{#1}}}}
\newcommand{\NormalTok}[1]{{#1}}

$if(natbib)$
\usepackage{natbib}
\bibliographystyle{$if(biblio-style)$$biblio-style$$else$plainnat$endif$}
$endif$
$if(biblatex)$
\usepackage[$if(biblio-style)$style=$biblio-style$,$endif$$for(biblatexoptions)$$biblatexoptions$$sep$,$endfor$]{biblatex}
$for(bibliography)$
\addbibresource{$bibliography$}
$endfor$
$endif$

% 该文档中首字符为“%”的均为注释行,不会在论文中出现

% 论文默认为双面模式,需单面模式请将第一行换为如下所示:
% \documentclass[oneside]{ZJUthesis}

% 取消目录中链接的颜色,方便打印
% 如需颜色,请将“false”改为“true”
\hypersetup{colorlinks=false}

\begin{document}
%%%%%%%%%%%%%%%%%%%%%%%%%%%%%
%% 正文字体设定
%%%%%%%%%%%%%%%%%%%%%%%%%%%%%
\fangsong

%%%%%%%%%%%%%%%%%%%%%%%%%%%%%
%% 论文封面部分
%%%%%%%%%%%%%%%%%%%%%%%%%%%%%
% 中文封面内容

% 中图分类号
\classification{$classification$}

% 单位代码
\serialnumber{$serialnumber$}

% 密级,如需密级则将其前“%”去掉
\SecretLevel{$SecretLevel$}

% 学号
\PersonalID{$PersonalID$}

\title{$titleshort$}
% 如果标题一行写不下,就写成两行,在下面的命令里写第二行,不需要两行则注释掉
% \titletl{一行写不下写两行}

%英文题目
\Etitle{$englishtitle$}
% 如果一行写不下,同中文题目设定,一行写不下则写两行,不需要就注释掉
% \Etitletl{The Second Line}

% 作者
\author{$author$}

\degree{$degree$}

% 导师
\supervisor{$supervisor$}

% 合作导师,如果有的话,去掉注释,
\cpsupervisor{$cpsupervisor$}

% 专业名称
\major{$major$}

% 研究方向
\researchdm{$researchdm$}

% 所属学院
\institute{$institute$}

%论文提交日期
\submitdate{$submitdate$}

% 答辨日期
\defenddate{$defenddate$}

% 生成封面
\makeCoverPage

%%%%%%%%%%%%%%%%%%%%%%%%%%%%%%
%% 中文题名页内容
%%%%%%%%%%%%%%%%%%%%%%%%%%%%%%
% 论文评阅人信息 注意两字名与三字名,两字职称与三字职称的写法,便于对齐
% 多余的名额直接注释掉即可,比如三个评阅人,把评阅人D,E注释掉即可
\reviewersA{$reviewersA$}
\reviewersB{$reviewersB$}
\reviewersC{$reviewersC$}
\reviewersD{$reviewersD$}
\reviewersE{$reviewersE$}

% 答辩委员会信息,如果某一个单位比较长,
% 请在其它较短后面补上{hspace{Xem}},X是比最长的单位名少几个字
% 如果实际人数少于6人,多余的注释掉即可
\chairman{$chairman$}
\commissionerA{$commissionerA$}
\commissionerB{$commissionerB$}
\commissionerC{$commissionerC$}
\commissionerD{$commissionerD$}
\commissionerE{$commissionerE$}

% 生成中文题名页
\maketitle


%%%%%%%%%%%%%%%%%%%%%%%%%%%%%%
%% 英文封面内容,硕士论文可不要此页
%%%%%%%%%%%%%%%%%%%%%%%%%%%%%%
% 英文题名
\englishtitle{$englishtitle$}
% 如果题名一行写不下,就写到第二行,不需要则将其注释掉
% \englishtitletl{The Second title Line}

% 评阅人信息,名字,职称,单位尽量用简写,否则会写不下
\EreviewersA{$EreviewersA$}
\EreviewersB{$EreviewersB$}
\EreviewersC{$EreviewersC$}
\EreviewersD{$EreviewersD$}
\EreviewersE{$EreviewersE$}

% 答辩委员会信息,同样尽量用简写,否则会写不下
\Echairman{$Echairman$}
\EcommissionerA{$EcommissionerA$}
\EcommissionerB{$EcommissionerB$}
\EcommissionerC{$EcommissionerC$}
\EcommissionerD{$EcommissionerD$}
\EcommissionerE{$EcommissionerE$}

% 生成英文封面
\makeenglishtitle


%%%%%%%%%%%%%%%%%%%%%%%%%%%%%%
%% 原创声明与版权协议页
%%%%%%%%%%%%%%%%%%%%%%%%%%%%%%

% 生成原创声明与版权协议页
\makeOSandCPRTpage


%%%%%%%%%%%%%%%%%%%%%%%%%%%%%%
%% 论文部分开始
%%%%%%%%%%%%%%%%%%%%%%%%%%%%%%
\ZJUfrontmatter



$body$


$if(natbib)$
$if(bibliography)$
$if(biblio-title)$
$if(book-class)$
\renewcommand\bibname{$biblio-title$}
$else$
\renewcommand\refname{$biblio-title$}
$endif$
$endif$
\bibliography{$for(bibliography)$$bibliography$$sep$,$endfor$}

$endif$
$endif$
$if(biblatex)$
\printbibliography$if(biblio-title)$[title=$biblio-title$]$endif$

$endif$

\end{document}
